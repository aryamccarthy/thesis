\chapter{INSULARITY AS A MEASURE OF NETWORK ROLES} \label{ch:insularity}% Must have a blank line after every section label

\section{Introduction}

\subsection{What even is a role?}

Guimer\`{a} and Amaral present the first model of identifying \say{universal} node \emph{roles} within communities, summarizing the information flow through each node relative to its community~\cite{guimera2005functional}.

~\cite{costa2007characterization}

\subsection{How did people find them before?}

\subsection{Centrality}

\subsection{Relationship between centrality and roles}

- Dolphins in the Newman paper

\section{Definition of insularity}

\subsection{What is insularity?}

\subsection{How does it relate to flowthrough centrality?}

\section{Visualization}

\subsection{How does it change at each level of the hierarchy?}

- Stejara's dendrograms again

\section{Experiment}

\subsection{What roles do we know about now?}

\subsection{Isn't it weird how some of our one-offs are very ``between'', while others are singletons?}

\section{Discussion}


%\section{Introduction} \label{sec:insularity introduction} % Must have a blank line after every section label
%
%Insularity is computed from the LP in a manner similar to Mann's computation of flowthrough centrality; however, it is a dynamic property of the network: a function of the network's utilization. 
%
%Flowthrough centrality is computed by solving the HMCFP to the point of total saturation; it is related to the node's the node's capacity (its weighted degree, a property of the graph) and the amount of flow terminating at that point, which is given in the solution of the HMCFP. 
%
%\section{Results}
%
%Stuff