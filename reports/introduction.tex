\chapter{INTRODUCTION} \label{ch:introduction}% Must have a blank line after every section label




% The area of research (hierarchical clustering and community detection)
This research illuminates the properties and comparative benefit of tools for community detection, or clustering, in networks. Community detection is of value because it is a method for unsupervised graph mining. 
% Most relevant previous findings (Mann, etc.)
Mann~\cite{mann2008sparsest} presented a community detection algorithm, the \emph{maximum concurrent flow (MCF) algorithm} that identifies the hierarchical structure of a network. At the same time, a number of other community detection algorithms emerged~\cite{lots of stuff}. Since the MCF algorithm was presented, new benchmarks for assessing the quality of community detection have been proposed~\cite{lancichinetti2008benchmark}. 

Research problem and why this is worthwhile studying

Objective of the thesis: how far you hope to advance knowledge of the field

Personal motivation: Why did you choose this topic?

Research method in brief: How will you find out?

Structure of the report: A \emph{paragraph} about each chapter. What is the main contribution of each chapter? How do they relate?

\section{Linear Programming}\label{sec:Linear Programming}

Mann's method relies on the maximum concurrent flow problem (MCFP), first investigated by Matula~\cite{matula1985concurrent}. This is a multicommodity flow problem with equal demand between all pairs of vertices in the network. In analogy to a minimum cut in a single-commodity flow problem, the critical saturated edges constitute a $k$-partite cut.

Mann presents the \emph{hierarchical} MCFP as the basis for clustering. The method partitions the network by a set of saturated edges. Next, it fixes the current demand between all pairs and selects the largest component to continue solving. 

~\cite{luenberger2008linear}

Operating with feature-based datasets incurs the additional complication of deciding the strategy for establishing presence of edges between each record in the feature space.

\section{Stability Analysis}\label{sec:Stability Analysis}

\section{Characterizing network roles}

\url{https://arxiv.org/pdf/cond-mat/0505185.pdf}

- vulnerability

- rich club coefficient

- degree distribution

\hrule

Spinglass is not hierarchical. Same with multilevel and Infomap and LPA.