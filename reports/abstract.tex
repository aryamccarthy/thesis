%Mann~\cite{mann2008sparsest} proposed a method of identifying the hierarchical community structure of a graph, called the \emph{MCF cut algorithm}. It iteratively solves the maximum concurrent flow problem (MCFP)~\cite{matula1985concurrent} to identify a sequence of sparsest cuts. This work explores the evolution of the MCF cut algorithm: a more stable algorithm which I refer to as the \emph{maximin clustering algorithm}. The aims of this work are to explore the stability of the maximin technique, to contextualize this method by comparison to other community detection (clustering) techniques in the literature, and to present an extension of both the MCF cut algorithm and the maximin algorithm to directed graphs. 
%Through simulations on LFR benchmark graphs, I show that \todo{Results.}. I prove that the maximin and MCF cut algorithms provide a truer representation of the hierarchical structure of communities than is possible with leading community detection algorithms, and that the maximin algorithm is more robust to perturbations in inputs than the MCF cut algorithm. Finally, I extend the problem instance and linear program solved for the MCFP at each stage of the two algorithms to handle diverse constraints on directed graphs.

%% TAKE 2 $$$$$$$$$$$$$$__----------________-------________------___--_--__-------_-___-----

%Community detection is the problem in network science of identifying tightly connected clusters defined by the topology of a network or graph. This differs from the problem of graph partitioning because the number of communities is unknown. The evaluation of a given partition of a graph into communities remains an open question; algorithms are typically tested on real-world or synthetic graphs with a known community structure. The objective is to maximize agreement between nodes' assignments to discovered and ground-truth communities.
%
%Communities often contain sub-communities, lending an additional dimension to the characterization of a network. These are evaluated similarly, on graphs with known or intentionally embedded hierarchical structure. 
%
%A maximin process is one that maximizes the minimum performance under some scoring criterion. In flow networks, the assignment of flow along paths between nodes can be a maximin process: the objective is to maximize the minimum flow between any pair of nodes in the network. The process necessarily saturates a set of edges with flow; these edges partition the graph into components. After this saturation is achieved, flow on each side of a partition can be increased to yield additional cuts and further partition the graph; this is the basis of the \emph{maximin algorithm} for community detection.

%Identifying the community structure and hierarchy of communities in a network is critical to identifying latent properties of the network. This task has real-world relevance in social network analysis, taxonomy, bioinformatics, and graph mining in general; nevertheless, evaluation is largely ad hoc and intuition-based. Popular algorithms struggle to recognize critical topological features. In this thesis, I analyze the \emph{maximin algorithm} for hierarchical community detection, which captures basic, intuitional notions of hierarchy that other methods do not. 

%I show that \todo{results}. I prove that the maximin and related \emph{MCF cut} algorithms provide a truer representation of the hierarchical structure of communities than is possible with leading community detection algorithms, and that the maximin algorithm is more robust to perturbations in inputs than the MCF cut algorithm. Finally, I formally define a version of the maximin and MCF cut algorithms for community detection on directed graphs.

Community detection (CD) is an important task in network science. Identifying the community structure and hierarchy of communities reveals latent properties of the network. This task has real-world relevance in social network analysis, taxonomy, bioinformatics, and graph mining in general. Nevertheless, there is no common definition of a community and no common, efficient method of identifying communities. As is common, we formulate CD as optimization of modularity. Modularity quantifies the separation of a network into distinct, highly interconnected groups. Maximizing modularity is NP-hard.

To solve the optimization problem, we present a polynomial-time approximation method. It greedily maximizes modularity with a heuristic for sparsest cuts in a network. This involves maximizing max-min fair throughput between all pairs of network nodes. We evaluate the approximation's effectiveness for CD on synthetic networks with known community structure. We show competitive results in terms of the standard measure of CD accuracy, normalized mutual information (NMI). Further, our method is less sensitive to network perturbations than existing community detection algorithms. Our method also detects ties in hierarchical structure, which other techniques do not.

In graphs without a strong community structure, our method does not impose arbitrary structure. In these cases, we can show that the max-min fair flow can be split onto edge-disjoint paths of a multigraph corresponding to the original network.

% Your abstract should begin with a definition of the problem you are solving, not the poor-man?s history of the algorithm you are basing your work upon. In fact, based on your abstract, it appears as if you will evaluate some random algorithm you?ve come up with (called maximin clustering algorithm) in the context of nothing?ok maybe providing a representation of hierarchical communities, whatever they are, even though there isn?t any problem being solved in those communities by your algorithm?at least no problems that you?ve actually identified?brutal enough for you?

% Start with a statement of the problem you are solving. Be sure to state your objective function ? which I assume is maximin? ? be sure to define maximin in your abstract .. then define your algorithm. �Then main results (note plural since this is thesis) and main conclusions and contributions (plural - should be at least three).

% Note, don?t say what you do?eg, don?t say ?I extend the problem instance and linear program ?.? �First, this is your abstract. Second, you?ve defined none of that, so it?s out of context, undefined and not understandable until after someone has read your thesis. �I?ll point you to my previous point - conclusions and contributions. �What you did is create an algorithm to solve a problem and then evaluated it. As far as your abstract is concerned, that?s all that matters as to what you did/how you did it.
